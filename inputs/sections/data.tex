	\section{Data.}

		Data represents the state of a content when a retrieval 
		request (GET) is performed and tells us if a content is 
		considered popular or not. Data comes from simulations
		using a threshold-based replication approach.
	
		In our data set, there are nine predictors/inputs and one 
		output/outcome. Table~\ref{table1} describes in details
		our data set and gives some examples of observed 
		values for inputs. 

		\begin{table}[h]
			\centering
			\begin{tabular}{ll}
				\hline
				Measurement &  Example \\
				\hline
				Content size in number of blocks &10 blocks \\
				Current number of replicas & 2 \\
				Active requests from gold customers  &  5 \\
				Active requests from silver customers &  1 \\
				Active requests from bronze customers &  4 \\
				Accomplished requests &  5 \\
				Time interval &  10 milliseconds \\
				Maximum average transfer rate  &  60 blocks per second\\
				Minimum average transfer rate  &  10 blocks per second \\
				\hline
			\end{tabular}
			\caption{Description of observed inputs for a content.}
			\label{table1}
		\end{table}

		%\href{http://lip6.fr/Guthemberg.Silvestre/resources/pcp/extras/content_popularity.data}{sample data set} 

		As outcome/output, we have a variable that has the value +1
		if the content is popular, otherwise -1. A 
		sample data set\footnote{Data set: \url{http://guthemberg.co.nr/resources/pcp/extras/content_popularity.data}} and 
		its info file\footnote{Info file: \url{http://guthemberg.co.nr/resources/pcp/extras/content_popularity.info}}
		are available online.  

		We might be able to collect information about the aggregate load
		on nodes that have replicas of a given content. 

%Table~\ref{table2} 
%		depicts this additional data.
%
%		\begin{table}[h]
%			\centering
%			\begin{tabular}{ll}
%				\hline
%				Measurement & Example \\
%				\hline
%				Number of contents being retrieved & 2 \\
%				Storage usage & 500 blocks \\
%				Transmitted data &  100 blocks \\
%				Duration &  50 seconds \\
%				Number of transfers &  20 \\
%				\hline
%			\end{tabular}
%			\caption{Additional inputs for a content.}
%			\label{table2}
%		\end{table}
		