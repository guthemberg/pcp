	\section{Introduction.}

		Multimedia content delivery on the Internet has changed dramatically in the recent
		years. Content distributed networks (CDNs) have allowed operators to
		provide content to the masses. Nowadays, ordinary users are able to 
		reach worldwide audiences thanks to web platforms, such as YouTube, deployed on top of CDNs.

		In order to handle the unpredictable load of online content on CDNs, computer system designers have provided
		mechanisms for adapting resource allocation according to the content's popularity.
		One of the most common mechanism is content replication.
		A good popularity-aware replication schemes contents should offer content replica 	
		maintenance to handle popularity growth properly.

		Broadly speaking, the aim of this study is predict the popularity of a content and
		perform replication accordingly. 

		Our current approach relies on transfer rate reservation and static thresholds. 
		Basically, it replicates a content whenever its bandwidth consumption reaches a fixed
		threshold. Through simulations with this approach, we found and published quite promising results. We are currently working in deploying and validate our finds through
		running experiments in a distributed testbed.
 
		As a future work, we plan to provide the similar mechanism that is driven by a statistical learning method. We consider using current simulations results as initial training data set of our learning model. 

		\clearpage 
 

